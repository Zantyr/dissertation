\documentclass[11pt]{article} %default 10pt
\usepackage[utf8]{inputenc}
\usepackage{geometry} % to change the page dimensions
\usepackage{booktabs} % for much better looking tables
\usepackage{verbatim} % adds environment for commenting out blocks of text & for better verbatim
\usepackage{float}
\usepackage{amsfonts}
\usepackage{ifpdf}
\usepackage{indentfirst}
\usepackage{fancyhdr} % This should be set AFTER setting up the page geometry
\usepackage[T1]{fontenc}
\usepackage[utf8]{inputenc}
\usepackage[polish]{babel}
\usepackage{caption}
\usepackage{mathtools}

\geometry{a4paper}
\geometry{margin=1in}
\pagestyle{plain}
\renewcommand{\headrulewidth}{0pt}
\lhead{}\chead{}\rhead{}
\lfoot{}\cfoot{\thepage}\rfoot{}
\setlength\parindent{24pt}

\usepackage{listings}
\usepackage{color}

\definecolor{dkgreen}{rgb}{0,0.6,0}
\definecolor{gray}{rgb}{0.5,0.5,0.5}
\definecolor{mauve}{rgb}{0.58,0,0.82}

\lstset{frame=tb,
    language=Python,
    aboveskip=3mm,
    belowskip=3mm,
    showstringspaces=false,
    columns=flexible,
    basicstyle={\small\ttfamily},
    numbers=none,
    numberstyle=\tiny\color{gray},
    keywordstyle=\color{blue},
    commentstyle=\color{dkgreen},
    stringstyle=\color{mauve},
    breaklines=true,
    breakatwhitespace=true,
    tabsize=3
}

\title{System doradztwa zawodowego z wykorzystaniem metod sztucznej inteligencji}
\author{Paweł Tomasik}

\begin{document}
\maketitle

\section{Spis treści}
Przepisać na Pythona 3.5
\section{Temat pracy}

\section{Klasyfikatory i klasteryzacja - rodzaje algorytmów i porównanie}
\subsection{Czym jest problem klasyfikacji i klasteryzacji}
\subsection{State-of-the-art metod sztucznej inteligencji}
\subsection{Binarny klasyfikator Bernoulliego}
\subsection{Sieć neuronowa}
\section{Badania zawodowe i psychologiczne}
\subsection{Obecne systemy klasyfikacji psychologicznej}
\subsection{Kryteria wyboru metod}
\subsection{Model ... (ten z sześcioma opcjami)}
\subsection{Model ... (z szesnastoma klasami)}
\subsection{Lista badanych zawodów}
\section{Wybór cech wejściowych}
\subsection{Problem ograniczenia ilości pytań}
\subsection{Okrojenie modeli psychologicznych}
\subsection{Dostosowanie pytań dodatkowych (data mining)}
\section{Implementacja i zbieranie danych}
\subsection{Moduł inteligentny}
\subsection{Moduł respondenta}
\subsection{Moduł uczenia wsadowego}
\subsection{Moduł prezentacji danych}
\subsection{Moduł kreacji testów}
\subsection{Aplikacja systemu do zadanego problemu}
\subsection{Schemat}
\section{Wyniki}
\subsection{Porównanie trafności klasyfikatorów}
\subsection{Wnioski wyciągnięte z badań}
\section{Kod źródłowy}
\subsection{Wsadowy skrypt instalacyjny (Unix)}
\subsection{Kod źródłowy modułu inteligentnego}
\lstinputlisting[breaklines]{src/main.py}

\begin{thebibliography}{9}
\bibitem{cichosz} Paweł Cichosz, \emph{Systemy uczące się}
\bibitem{2} ..., \emph{Książka od Łukasza Dzianacha}
\bibitem{rutkowski} Leszek Rutkowski, \emph{Metody i techniki sztucznej inteligencji}, Warszawa 2005, PWN, rozdziały 6, 8, 10
\bibitem{szczepaniak} Piotr S. Szczepaniak \emph{Obliczenia inteligentne, szybkie przekształcenia i klasyfikatory}, Warszawa 2004, Akademicka Oficyna Wydawnicza EXIT
\bibitem{wojcik} Waldemar Wójcik et alii, \emph{Sztuczna inteligencja i metody optymalizacji - od teorii do praktyki}, Lublin 2008, Polskie Towarzystwo Informatyczne
\bibitem{parol} Mirosław Parol, Paweł Piotrowski et alii, \emph{Sztuczna inteligencja w praktyce - Laboratorium}, ćwiczenie 3
\bibitem{malina} Witold Malina, Maciej Smiatacz, \emph{Rozpoznawanie obrazów}, Warszawa 2010, akademicka Oficyna Wydawnicza EXIT
\bibitem{podolak} Igor T. Podolak, \emph{Klasyfikator Hierarchiczny z nakładającymi się grupami klas}, Kraków 2012, Wydawnictwo Uniwersytetu Jagiellońskiego
\end{thebibliography}
\end{document}
